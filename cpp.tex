 \documentclass[11pt,final]{book}
 \title{The Rook's Guide to C++}
 \date{26 November 2013}

%%%%%%%%%%%%%%%%%%%%%%%%%%%%%%%%%%%%%%%%%%%%%%%%%%%%%%%%%%%%%%%%%%%%%%%
%%                                                                    %
%% 2013-01-30:                                                        %
%%   To get the proper thumbs up:                                     %
%%     Rename the checkmark in the dingbats.sty file to, um,          %
%%  dingbatcheckmark, then use this name in the \newcommand           %
%%                                                                    %
%%%%%%%%%%%%%%%%%%%%%%%%%%%%%%%%%%%%%%%%%%%%%%%%%%%%%%%%%%%%%%%%%%%%%%%
\hyphenation{Je-re-my}
\hyphenation{Le-Blanc}
\hyphenation{Ver-gnes}
\hyphenation{Zach-a-ry}
\hyphenation{Pat-rick}
\hyphenation{Du-Harte}
\hyphenation{Ry-an}
\hyphenation{Mat-thew}
\hyphenation{Ke-vin}
\hyphenation{Wer-zing-er}
\hyphenation{Mi-chael}
\hyphenation{Ar-chie}
\hyphenation{Bo-ren-stein}
\hyphenation{Buck-ley}
\hyphenation{Ca-ma-ra}
\hyphenation{Mac-Gil-liv-ray}
\hyphenation{Li-ber}
\hyphenation{Stu-art}
\hyphenation{Ma-rone}
\hyphenation{maz-zel-lo}
\hyphenation{Mo-ra-les}
\hyphenation{Tor-res}
\hyphenation{Ne-ben-fuhr}
\hyphenation{Pat-rice}
\hyphenation{Dan-i-el}
\hyphenation{Ta-ba-ry}
\hyphenation{De-vin}
\hyphenation{Wil-li-am-i-tis}
\hyphenation{To-ny}
\hyphenation{A-dam}
\hyphenation{Wheel-er}
\hyphenation{Lo-thar}
\hyphenation{Sand-berg}
\hyphenation{Val-de-mar}
\hyphenation{Da-vies}
\hyphenation{Jo-seph}
\hyphenation{Wil-li-am}
\hyphenation{Mad-sen}
\hyphenation{Carl-ton}
\hyphenation{Do-mi-nic}
\hyphenation{Ash-ley}
\hyphenation{Da-vid}
\hyphenation{Cris-to-fo-ri}
\hyphenation{Mes-ki-men}
\hyphenation{Bran-don}
\hyphenation{Nord-garden}
\hyphenation{A-leks-an-der}
\hyphenation{Pa-da-wer}
\hyphenation{Aar-on}
\hyphenation{Mc-In-tosh}
\hyphenation{Moul-ton}
\hyphenation{niel-sen}
\hyphenation{Pon-tus}
\hyphenation{Nils-son}
\hyphenation{Mar-shall}
\hyphenation{To-ny}
\hyphenation{Wil-li-ams}
\hyphenation{Dy-lan}
\hyphenation{Wi-dis}
\hyphenation{Ste-phen}
\hyphenation{Ja-net}
\hyphenation{Maw-yer}
\hyphenation{Phi-lip}
\hyphenation{Pet-rov}
\hyphenation{Slon-ka}
\hyphenation{Til-brook}
\hyphenation{Ar-man-do}
\hyphenation{Pin-ches}
\hyphenation{Ring-man}
\hyphenation{E-man-u-el}
\hyphenation{Pe-ter-son}
\hyphenation{Rich-ard}
\hyphenation{Ko-gut}
\hyphenation{Mitch-ell}
\hyphenation{Sig-mund}
\hyphenation{Sat-tler}
\hyphenation{Le-vi}
\hyphenation{Ra-man}
\hyphenation{Ko-soy}
\hyphenation{Ja-mie}
\hyphenation{cstdlib}
\hyphenation{NetBeans}
\hyphenation{getline}
\hyphenation{setBase}

%\usepackage{arev}             % Nice heart. ($\varheart$)
%\usepackage{dingbat}          % Fists (thumbs up} (better)
%\usepackage{bbding}           % Fists (thumbs up}
%\usepackage{yfonts}           % German font(s).
%\usepackage{ulem}             % strikeout fonts.

%\newcommand{\Frak}            {\textfrak}
%\newcommand{\Frak}            {\textgoth}
%\newcommand{\Frak}            {\textswab}

%\newcommand{\F}[1]            {\Frak{#1}}

\usepackage{libertine}
\usepackage{fontspec}
\usepackage{tablefootnote}
%\usepackage{polyglossia}
%\usepackage{xunicode}
%\usepackage{xltxtra}

% Packages loaded by todonotes.  Included here so that needed options may be specified.
\usepackage{ifthen}
%\usepackage{xkeyval}
\usepackage{xcolor}
%\usepackage{tikz}
%\usepackage{calc}
\usepackage{graphicx}         % Loaded via the tikz package.
%\usepackage{todonotes}        % http://www.tex.ac.uk/tex-archive/macros/latex/contrib/todonotes/todonotes.pdf

 %\definecolor{MainFont}        {rgb}{0.000000, 0.180000, 0.180000}
 
 %\setmainfont[
 %  Color=MainFont,
 %  Opacity=0.0,
 %  Ligatures=TeX
 %  ]
 %  {Linux Biolinum}
%  {DejaVu Serif}

 %\definecolor{SansFont}        {rgb}{0.000000, 0.080000, 0.080000}
% \setsansfont[
%   Color=SansFont,
%   Opacity=0.0,
%   Ligatures=TeX
%   ]
%   {Linux Biolinum}
%  {DejaVu Sans}
 \usepackage{verbatim}
%\usepackage{wasysym}

 \usepackage{framed}
 \usepackage{layout}
\usepackage[paperwidth=6.0in, paperheight=9.0in]{geometry}
%\usepackage[paperwidth=8.5in, paperheight=11.0in]{geometry}
 \usepackage[pdfauthor={Jeremy A. Hansen, Ph.D.}, pdftitle={The Rook's Guide to C++}]{hyperref}
 \urlstyle{same}
%%%%%%%%%%%%%%%%%%%%%%%%%%%%%%%%%%%%%%%%%%%%%%%%%%%%%%%%%%%%%%%%%%%%%%%
%%                                                                   %%
%%   The listings package produces nicely-formatted source-code      %%
%% listings.                                                         %%
%%                                                                   %%
%% http://mirrors.ibiblio.org/CTAN/macros/latex/contrib/listings     %%
%%                                                                   %%
%%   We specified only the parameters necessary produce this         %%
%% document.  There are more.  Consult the package documentation     %%
%% at the above site.                                                %%
%%                                                                   %%
%% Wikibooks has a writeup, too.                                     %%
%% \href{http://en.wikibooks.org/wiki/LaTeX/Source_Code_Listings}    %%
%%     {source code listings package}                                %%
%%                                                                   %%
%%%%%%%%%%%%%%%%%%%%%%%%%%%%%%%%%%%%%%%%%%%%%%%%%%%%%%%%%%%%%%%%%%%%%%%

 \usepackage{listings}

 \usepackage{fancyhdr}
 \setlength{\headheight}{15.2pt}

 \usepackage{makeidx}
 \makeindex
 \raggedbottom
 
 \input{utils-commands.tex}


 \setcounter{tocdepth}{4}

 \begin{document}
 \maketitle
 \thispagestyle{empty}
 \newpage

 \setcounter{page}{1}
 \pagenumbering{roman}
% \pagestyle{plain}

 \thispagestyle{empty}
%% copyrightpage
\begingroup
\normalsize
\parindent 0pt
\parskip \baselineskip
\textcopyright{} 2013 Jeremy A. Hansen \\
All rights reserved.

    This work is licensed under a Creative Commons Attribution-NonCommercial-ShareAlike 3.0 Unported License, as described at 

\noindent \url{http://creativecommons.org/licenses/by-nc-sa/3.0/legalcode}

Printed in the United States of America

\vfill

\begin{center}
\begin{tabular}{ll}
First edition:  & November 2013 \\
\end{tabular}
\end{center}

\vfill

ISBN 978-1-304-66105-0

\vfill

Rook's Guide Press \\
19 Black Road \\
Berlin, VT 05602 \\
\url{http://rooksguide.org}

%%%%{\LARGE\plogo}
\vspace*{2\baselineskip}


\endgroup
%\clearpage

\newpage

 \setcounter{page}{1}


\chapter*{Preface\markboth{\MakeUppercase{Preface}}{}}

What you are reading is the first of what I hope to be many ever-improving iterations of a useful C++ textbook.
We've gone fairly quickly from whim to print on an all-volunteer basis, and as a result, there are many things that I'd add and change if I had an infinite amount of time in my schedule.
The vast majority of the contents were written in less than 36 hours by 25 students (mostly freshmen!) at Norwich University over a long weekend.
Some of it is mine, and some was added by our crack team of technical editors as we translated sleep-deprived poor grammar into sleep-deprived better grammar.

Where it goes from here is mostly up to you!
If there's a section that's missing or in need of clarification, please take a bit of time and make those changes.
If you don't want to bother yourself with the GitHub repository, send me your additions and modifications directly.

I want to first thank my family for the time I didn't spend with them on the writing weekend and throughout the summer when I was editing and typesetting. I promise I won't do this next summer!

My next thanks go out to the technical editors and typesetters, without whom you would have a much uglier book.
Thanks to Ted Rolle for building the initial \LaTeX framework and to Matt Jadud for the incredibly helpful pointers on how to manage the pile of typesetting files in a collaborative environment. I also thank Craig Robbins and Levi Schuck, who, on different sides of the planet, managed to contribute extensively to the heavy lifting of getting the book into the shape it's in now. If we ever meet, I owe you a beer or whatever you're having!

I also would like to thank all of the Kickstarter backers not only for the money which made this possible, but for reinforcing the idea that this is a worthwhile contribution to the community. Peter Stephenson and Andrew Pedley also contributed food directly over the textbook writing hackathon weekend, and without them we'd never have gotten our saturated fat quota! (Note to future project leaders: there's nothing that gets a bunch of college students who are generally lukewarm about programming to write a textbook like free food. It didn't even matter what the food was. Really.)

Thanks to Matt Russo for shooting the video and organizing the media and social networking efforts with the Kickstarter project through the writing weekend.

Special thanks to Allyson LeFebvre\footnote{That's ``la-fave'', everyone} for the textbook photography, several diagrams, and the extensive search through the semi-final textbook that turned up a bunch of mistakes that I missed.

And my last (and not at all least) thanks go out to all the students who showed up in person or digitally. And without getting too grandiose, you remind us all that we can make the world better by showing up. Keep showing up!

~ \linebreak

\noindent \textbf{Jeremy}

\noindent \texttt{jeremyhansen@acm.org}

\noindent 26 November 2013

~ \linebreak

\pagebreak 

 \tableofcontents

%%%%%%%%%%%%%%%%%%%%%%%%%%%%%%%%%%%%%%%%%%%%%%%%%%%%%%%%%%%%%%%%%%%%%%%
%%                                                                   %%
%%   Here are the listing package's parameters used in the creation  %%
%% of this book.                                                     %%
%%                                                                   %%
%%%%%%%%%%%%%%%%%%%%%%%%%%%%%%%%%%%%%%%%%%%%%%%%%%%%%%%%%%%%%%%%%%%%%%%
\input{utils-listings.tex}


 %\Comment{ % Frontmatter.

\chapter*{License\markboth{\MakeUppercase{License}}{}}


\vspace{1in}

\includegraphics[width=.25\textwidth]{graphics/cc.large.png} \
\includegraphics[width=.25\textwidth]{graphics/by.large.png} \
\includegraphics[width=.25\textwidth]{graphics/nc.large.png} \
\includegraphics[width=.25\textwidth]{graphics/sa.large.png}

\vspace{1in}



\noindent This work by Jeremy A. Hansen (jeremyhansen@acm.org) is licensed under a Creative Commons Attribution-NonCommercial-ShareAlike 3.0 Unported License, as described at \newline

\noindent \footnotesize \url{http://creativecommons.org/licenses/by-nc-sa/3.0/legalcode}





% \\A \href{http://creativecommons.org/licenses/by-nc/3.0/}


\chapter*{Dramatis Person\ae\markboth{\MakeUppercase{Dramatis Person\ae}}{}}

 \begin{description}

 \item[Managing Editor:] ~
 
 Jeremy A. Hansen, PhD, CISSP

 \item[Technical Editing \& Typesetting:] ~
 
 Jeremy A. Hansen
 
 Matt Jadud, PhD
 
 Craig D. Robbins
 
 Theodore M. Rolle, Jr.
 
 Levi Schuck

 \item[Media \& Outreach:] ~
 
 Matthew E. Russo

 \item[Cover Art \& Graphic Design:] ~
 
 Allyson E. LeFebvre

 \item[Content Authors:]\label{ContentAuthors} ~
 
Tyler Atkinson,
Troy M. Dundas,
Connor J. Fortune,
Jeremy A. Hansen,
Scott T. Heimann,
Benjamin J. Jones,
Michelle Kellerman,
Michael E. Kirl,
Zachary LeBlanc,
Allyson E. LeFebvre,
Gerard O. McEleney,
Phung P. Pham,
Megan Rioux,
Alex Robinson,
Kyle R. Robinson-O'Brien,
Jesse A. Rodimon,
Matthew E. Russo,
Yosary Silvestre,
Dale R. Stevens,
Ryan S. Sutherland,
James M. Verderico,
Christian J. Vergnes,
Rebecca Weaver,
Richard Z. Wells, 
Branden M. Wilson and
Robert J. Garner.

 \item[Funding \& Support:] ~
 
Peter Stephenson, PhD, VSM, CISSP, CISM, FICAF, LPI at the Norwich University Center for Advanced Computing \& Digital Forensics

Andrew Pedley at Depot Square Pizza
 \end{description}

\noindent \textbf{Kickstarter contributors:}   

\noindent \input{kickstarters.tex}

% } % End Frontmatter Comment.

% \LevelA{Section 1}
   %\LevelB{Chapters:}
      \LevelC{History}
 \setcounter{page}{1}
 \pagenumbering{arabic}
 %\pagestyle{fancy}
			\label{chap_history}
      \input{chap_history.tex}
%      \LevelC{Types of languages}
%      \input{chap_types.tex}
%      \LevelC{How Computers Use Memory}
%      \input{chap_memory.tex}
%      \LevelC{Sample Program}
%			\label{chap_sampleprogram}
%      \input{chap_sampleprogram.tex}
% \LevelA{Section 2}
   %\LevelB{Chapters:}
     \LevelC{Development Environment}     
      \label{chap_devenvironment}
      % This work by Jeremy A. Hansen is licensed under a Creative Commons 
% Attribution-NonCommercial-ShareAlike 3.0 Unported License, 
% as described at http://creativecommons.org/licenses/by-nc-sa/3.0/legalcode


In order to write code we must have a development environment. There are many sollutions to do this. Some can be as simple as having a text editor and a command line compiler and linker program. Another solution is to install a Integrated Development Environment (IDE). One of the most popular IDE's for C++ development is Visual Studio Community. Visual Studio Community is free and can be downloaded from the web. For those who can not use visual studio community we will also explain how to use Visual Studio Code which is also free but runs on Windows, Mac and Linux. In this chapter we will explain how to install Visual Studio Community or Visual Studio Code as an IDE and will provide you instruction on how to make a basic Hello World Application.



	
     \LevelC{Variables}
			\label{chap_variables}
      \input{chap_variables.tex}
     \LevelC{Literals and Constants}
			\label{chap_constants}
      \input{chap_constants.tex}
     \LevelC{Assignments}
			\label{chap_assignments}
      \input{chap_assignments.tex}
     \LevelC{Output}
			\label{chap_output}
      % This work by Jeremy A. Hansen is licensed under a Creative Commons 
% Attribution-NonCommercial-ShareAlike 3.0 Unported License, 
% as described at http://creativecommons.org/licenses/by-nc-sa/3.0/legalcode

Output in C++ is done with the object \Code{cout} (``\textbf{c}onsole \textbf{out}put''). 
The object \Code{cout} prints useful information to the screen for the user. 
For example, if we wanted to prompt the user with 

\noindent \Code{Type in your name:}

\noindent we would use \Code{cout}. 
\Code{cout} is extremely important when you are starting to learn C++ as it gives you the ability to display the current state of any variable and provide user feedback at any point in your program. 
Let's make a program that outputs something to the screen:

\noindent\begin{minipage}{\linewidth}\begin{lstlisting}
#include <iostream>
using namespace std;
int main()
{
  cout << "Go Cadets!";
  return 0;
}
\end{lstlisting}\end{minipage}

%Don't mind the grayed out code, that's necessary, but we'll get to it later. 
%Your development environment may provide much of that code anyways; if not, type it in, but don't worry about that material just yet. 

The symbol \Code{<<} is called the \Keyword{insertion operator} and is needed between \Code{cout} and what you want to display to the screen. 
In this case, we are displaying a string literal \Code{"Go Cadets!"}. 
As you know, every statement in C++ ends with a semicolon, and this one is no exception.

What if we want to print more, though?

\noindent\begin{minipage}{\linewidth}\begin{lstlisting}
#include <iostream>
using namespace std;
int main()
{
  cout << "Go Cadets!";
  cout << "You can do it!";
  return 0;
}
\end{lstlisting}\end{minipage}

Try to compile and run that. 
It works, but it's not really the desired output. 
You should get:

\noindent \Code{Go Cadets!You can do it!}

How do we get those on a different line? 
One of the ways we can do it is to use the object \Code{endl}. 
\Code{endl} means ``\textbf{end} \textbf{l}ine'', and is used when you want to end one line and start over on the next---it's like hitting enter on your keyboard. 
You will also need another redirect operator between the string literal and the \Code{endl}. 
Putting all of this together looks like this:

\noindent\begin{minipage}{\linewidth}\begin{lstlisting}
#include <iostream>
using namespace std;
int main()
{
  cout << "Go Cadets!" << endl;
  cout << "You can do it!";
  return 0;
}
\end{lstlisting}\end{minipage}

\noindent This prints:

\noindent \Code{Go Cadets!}

\noindent \Code{You can do it!}

That works a bit more as intended. 
Alternatively, we can combine the two lines that use \Code{cout} into a single one like this:

\noindent\begin{minipage}{\linewidth}\begin{lstlisting}
cout << "Go Cadets!" << endl << "You can do it!";
\end{lstlisting}\end{minipage}

Another way we can accomplish this, without needing another redirect operator, is with the special character \Code{'\textbackslash n'}.
\Code{'\textbackslash n'} is a newline character, it prints a new line just like the \Code{endl} object. 

\noindent\begin{minipage}{\linewidth}\begin{lstlisting}
#include <iostream>
using namespace std;
int main()
{
  cout << "Go Cadets!\nYou can do it!";
  return 0;
}
\end{lstlisting}\end{minipage}

\noindent This prints:

\noindent \Code{Go Cadets!}

\noindent \Code{You can do it!}

Another thing we can use with the console output object is the special character \Code{'\textbackslash t'}. 
Printing this character is the same as pressing the tab key on your keyboard, and is used for indentation and formatting. 
Let's look at an example that uses the newline character, the tab character, and some text:

\noindent\begin{minipage}{\linewidth}\begin{lstlisting}
#include <iostream>
using namespace std;
int main()
{
  cout << "\tGo Cadets!\nYou can do it!";
  return 0;
}
\end{lstlisting}\end{minipage}

\noindent This code prints: 

\Code{Go Cadets!}

\noindent \Code{You can do it!}

We don't always have to output words the screen using \Code{cout}. 
We can also print variables of type \Code{int}, \Code{double}, and \Code{float} and can control the number of digits that appear after the decimal point. 
For example, if we had a variable that contained the value 3.14159265 we might only care about the first two numbers after the decimal point and just want to output 3.14 to the screen.
Note that there is also one digit before the decimal point.
We do that with the \Code{precision()} member function. 
This function call will result in subsequent \Code{float} or \Code{double} variables being printed with the specified number of decimal places. 
In the following code, the number of digits is set to 3:

\noindent\begin{minipage}{\linewidth}\begin{lstlisting}
#include <iostream>
using namespace std;
int main()
{
  double num = 3.14159265;
  cout.precision(3);
  cout << num << endl;
}
\end{lstlisting}\end{minipage}

\noindent This code prints:

\noindent \Code{3.14}

To display data in a similar way as a spreadsheet, we can create a field of characters and set the number of characters in each field using the \Code{width()} and \Code{fill()} member functions. 
Notice the use of the \Code{left} flag in the following code, which positions the output on the left side of the field; the default is for the output to be on the right side:

\noindent\begin{minipage}{\linewidth}\begin{lstlisting}
#include <iostream>
using namespace std;
int main()
{
  cout << "Norwich" << endl;
  cout.width(15);
  cout << "University" << endl;
  cout.fill('*');
  cout.width(20);
  cout << left << "Corps of Cadets" << endl;
}
\end{lstlisting}\end{minipage}

\noindent The above code prints:

\noindent \Code{Norwich}

\noindent \Code{~~~~~University}

\noindent \Code{Corps of Cadets*****}

\LevelD{Review Questions}
\begin{enumerate}
	\item Which of the following is a correct way to output \Code{Hello World} to the screen?
	  \begin{enumerate}
	  \item \Code{output: "Hello World";}
	  \item \Code{cout >> "Hello World";}
	  \item \Code{cout << "Hello World";}
	  \item \Code{console.output << "Hello World";}
	  \end{enumerate}
  \item Which of the following is a correct way to output \Code{Hello!} to the screen on one line and \Code{Goodbye!} to the screen on the next line?
		\begin{enumerate}
		\item \Code{cout >> "Hello!" >> "Goodbye!";}
		\item \Code{output: "Hello!\textbackslash nGoodbye!";}
		\item \Code{cout << "Hello!" << \textbackslash n << "Goodbye!";}
		\item \Code{cout << "Hello!" << '\textbackslash n' << "Goodbye!";}
		\end{enumerate}
  \item Aside from the answer in the previous question, write two alternative ways of printing \Code{Hello!} and \Code{Goodbye!} to the screen on two different lines.
	\item Write several lines of code using the \Code{width()} and \Code{fill()} functions in a \Code{main()} that prints \Code{Programming!} to the screen with 4 \Code{'x'} characters printed after it.
	\item Write code to output the values 124, 12.376, \Code{z}, 1000000, and \Code{strings!} as distinct values, separated by spaces.
	\item What is the output of the following program?

\noindent\begin{minipage}{\linewidth}\begin{lstlisting}
#include <iostream>
#include <string>
using namespace std;
int main()
{
  string shirt = "maroon", pants = "blue";

  cout << shirt << " " << pants << endl;
  return 0;
}
\end{lstlisting}\end{minipage}

\end{enumerate}

\LevelD{Review Answers}
\begin{enumerate}
	\item c.
	\item d.
	\item \Code{cout << "Hello!" << endl << "Goodbye!";} or
	
	\Code{cout << "Hello!\textbackslash nGoodbye!";} 
	
	(other similar answers are possible)
	\item
\noindent\begin{minipage}{\linewidth}\begin{lstlisting}
cout.fill('x');
cout.width(16);
cout << left << "Programming!";
\end{lstlisting}\end{minipage}
	\item \Code{cout << 124 << " " << 12.376 << " z " << 1000000 <<  " strings!";}
	\item \Code{maroon blue}
\end{enumerate}

\LevelD{Further Reading}

\begin{itemize}
\item \url{http://java-samples.com/showtutorial.php?tutorialid=245}
\item \url{http://www.cplusplus.com/doc/tutorial/basic_io}
\item \url{http://www.cplusplus.com/reference/ostream/ostream/}
\item \url{http://www.cplusplus.com/doc/tutorial/functions/}
\end{itemize}	

     \LevelC{Input}
			\label{chap_input}
      \input{chap_input.tex}
     \LevelC{Arithmetic}
			\label{chap_arithmetic}
      \input{chap_arithmetic.tex}
     \LevelC{Comments}
			\label{chap_comments}
      \input{chap_comments.tex}
     \LevelC{Data Types and Conversion}
			\label{chap_datatypes}
      \input{chap_datatypes.tex}

% \LevelA{Section 3}
%   \LevelB{Chapters:}
     \LevelC{Conditionals}
			\label{chap_conditionals}
      \input{chap_conditionals.tex}
     \LevelC{Strings}
			\label{chap_strings}
      \input{chap_strings.tex}
     \LevelC{Loops}
      \label{chap_loops}
      \input{chap_loops.tex}
     \LevelC{Arrays}
			\label{chap_arrays}
      \input{chap_arrays.tex}
     \LevelC{Blocks, Functions, and Scope}
			\label{chap_functions}
      \input{chap_functions.tex}
     \LevelC{Problem Solving \& Troubleshooting}
			\label{chap_problems}
      \input{chap_problems.tex}
%     \LevelC{Testing}
%			\label{chap_testing}
%      \input{chap_testing.tex}

% \LevelA{Section 4}
%   \LevelB{Chapters}
     \LevelC{The Preprocessor}
			\label{chap_preproc}
      \input{chap_preproc.tex}
     \LevelC{Advanced Arithmetic}
			\label{chap_advancedarith}
      \input{chap_advancedarith.tex}
     \LevelC{File I/O}
			\label{chap_file_io}
      \input{chap_file_io.tex}
     \LevelC{Pointers}
			\label{chap_pointers}
      \input{chap_pointers.tex}
     \LevelC{Dynamic Data}
			\label{chap_dynamic}
      \input{chap_dynamic.tex}

 \Comment{ % LevelX comment.

 \LevelA{}
   \LevelB{}
     \LevelC{}
     \LevelC{}
     \LevelC{}
     \LevelC{}
     \LevelC{}
     \LevelC{}
     \LevelC{}
     \LevelC{}
 } % End LevelX comment.

% \LevelA{Section 5}
%   \LevelB{Chapters}
%     \LevelC{User-defined types}
%      \input{chap_userdefined.tex}
     \LevelC{Classes and Abstraction}
			\label{chap_classes}
      \input{chap_classes.tex}
%     \LevelC{Exceptions}
%      \input{chap_exceptions.tex}
     \LevelC{Separate Compilation}
			\label{chap_separate}
      \input{chap_separate.tex}
     \LevelC{The C++ Standard Library}
			\label{chap_stl}
      \input{chap_stl.tex}

 %\missingfigure{Pictures and figures and tables are a nice touch.}

 %\listoftodos

 %\lstlistoflistings

 \printindex

 \end{document}
